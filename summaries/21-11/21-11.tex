\documentclass[11pt]{article}

\usepackage[top=1in, bottom=1in, left=1in, right=1in]{geometry}
\usepackage{bm}
\usepackage{amsmath}
\usepackage{tikz}
\usepackage{xcolor}
\usepackage{bbold}

\newcommand{\unit}[1]{\hat{\mathbf{#1}}}
\newcommand{\parens}[1]{\left( #1 \right)}
\newcommand{\brackets}[1]{\left[ #1 \right]}
\newcommand{\jtd}[1]{{\color{red}\textit{#1}}}
\usetikzlibrary{arrows.meta}

\begin{document}
\title{Asteroid project summary}
\author{Jack Dinsmore, Julien de Wit}

\maketitle

\section{Introduction}


\section{Asteroid model}
In an effort to make a general asteroid model, we will consider the entire effect of the tidal torque applied to an asteroid from a planet, rather than solving for the first order perturbation. To better understand the effect, we will not model perturbing effects from other bodies such as the Sun or other planets. However, nearby bodies, such as moons or rings, are considered in a later section. We will also model the asteroid as a rigid body, whose density distribution is fixed throughout the flyby.

Most asteroids are observed to have attained their minimum energy rotation state, so we will also assume that the asteroid's initial state aligns its rotational velocity parallel to the principal axis with maximal moment of inertia.

\subsection{Coordinates}
\label{sec:coordinates}
We will make use of two frames of reference to model this system. One is the ``inertial frame,'' with axes denoted by $\unit{X}$, $\unit{Y}$, $\unit{Z}$. As the name suggests, these axes are inertial, with $\unit{X}$ pointing to the asteroid pericenter and $\unit{Z}$ pointing parallel to the orbit angular momentum. The origin of this frame is set to the center of mass of the central body. We will assume that the mass distribution of the central body is known in this inertial frame.

Our second frame is the ``body-fixed'' frame, denoted by $\unit{x}, \unit{y}, \unit{z}$. This frame is fixed with respect to the body's principle axes and rotates with the body, with its origin at the body's center of mass. We will solve for the asteroid's mass distribution with reference to the body-fixed frame. For definiteness, we define $\unit{z}$ to be the principal axis with maximal MOI, and $\unit{x}$ to have minimal MOI.

We define a rotation matrix $M$ such that $M\bm{r} = \bm{R}$, where $\bm{r}$ is in the body-fixed frame and $\bm{R}$ is in the inertial frame. It will be useful for us to represent this matrix in two different ways; one is with a quaternion $\bm q$, and the other is with $z-y-z$ Euler angles named $\alpha$, $\beta$, and $\gamma$. In particular, we define $M = R_z(\gamma) R_y(\beta) R_z(\alpha)$ where $R_i(\theta)$ is a rotation around the $i$th unit vector by $\theta$.

The Euler angles are necessary for our calculations, but we do not use them as dynamical variables because they suffer from gimbal lock. Instead, we use the quaternion $\bm q$

\begin{figure}
\centering
\begin{tikzpicture}
\draw[-{Latex[length=3mm]}] (0, 0) -- (-4, 0) node[anchor=east] {$\unit x$};
\draw[-{Latex[length=3mm]}] (0, 0) -- (2, -3) node[anchor=west] {$\unit y$};
\draw[-{Latex[length=3mm]}] (0, 0) -- (0, 4) node[anchor=south] {$\unit z$};

\draw[dashed, -{Latex[length=3mm]}] (0, 0) -- (3.7, -2) node[anchor=north] {$\unit n$};

\draw[thick,-{Latex[length=3mm]}] (0, 0) -- (-0.5, -3) node[anchor=north] {$\unit X$};
\draw[thick,-{Latex[length=3mm]}] (0, 0) -- (4, 0.5) node[anchor=south] {$\unit Y$};
\draw[thick,-{Latex[length=3mm]}] (0, 0) -- (-2, 3) node[anchor=south] {$\unit Z$};

\draw (0.5, -0.75) arc (290:302:2.5);
\draw (0.9, -0.9) node[anchor=center] {$\alpha$};

\draw (0, 1) arc (90:123.7:1);
\draw (-0.4, 1.2) node[anchor=center] {$\beta$};

\draw (0.97, -0.52) arc (330:360:1.3);
\draw (1.4, -0.25) node[anchor=center] {$\gamma$};

%\draw[-{Latex[length=3mm]}] (0, 0) -- (0, 4) node[anchor=south] {$\unit Z$};
%\draw[-{Latex[length=3mm]}] (0, 0) -- (4, 0) node[anchor=west] {$\unit Y$};
%\draw[-{Latex[length=3mm]}] (0, 0) -- (-2, -2) node[anchor=east] {$\unit X$};


\end{tikzpicture}
\caption{$z-y-z$ Euler angles used in this work to express the orientation of the asteroid. Orientation is expressed as a rotation from the body-fixed axes to the inertial axes.}
\label{fig:euler-angles}
\end{figure}


\section{Experiment design}


\section{Results}


\section{Uncertainty testing}


\section{Conclusion}

\bibliographystyle{unsrt}
\bibliography{asteroids}

\end{document}
